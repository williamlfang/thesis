% !Mode:: "TeX:UTF-8"

\chapter{绪论}{Introduction}
\label{chap01}
利率是经济和金融领域的核心变量,不仅是学术界研究与讨论的重要话题,而且长期受到政府政策部门与市场投资者的广泛关注。利率反映了宏观经济运行状态、商业周期、货币与财政政策影响效果、社会变化和人们对未来的预期,是连接实际经济因素与货币因素的中介变量,在数量上揭示了货币信贷市场的不同期限的资金供求关系,是各国中央银行用来控制短期利率进而影响中长期利率变化和改善实体经济的主要货币政策传递机制;同时,利率还是金融市场中各类资产定价、金融产品设计、利率风险管理与套期保值的基准依据,如著名的~Black-Scholes~期权定价公式以无风险利率作为基准参数输入。由于利率的随机性特征,利率建模也因此在经济与金融领域成为一个最为复杂和最富挑战性的研究热点\citep{gibson2010modeling}。

\section{研究背景与研究动机}{Background and Motivation}

利率期限结构(Term Structure of Interest Rates)是用来描述某一时点上,不同到期日债券的到期收益率与到期期限之间关系的数学模型,在图形上则表现为一系列可观察的曲线族。对于零息票国债券的利率期限结构,也称作`` 收益率曲线''(Yield Curve)。一方面,\ts 受宏观经济与金融领域多种因素的影响,整个经济系统中的各种因素都直接或间接地通过货币市场与金融市场因素影响债券市场,从而影响\ts 的形成机制。典型的收益率曲线形状包括四种,即上升的收益率曲线、向下倾斜的收益率曲线、呈拱形的收益率曲线和平坦的收益率曲线。从短期上看,\yc 能够捕捉到几乎所有的宏观经济指标的变动情况\citep{ang2003no,monch2008forecasting},从而为中央银行制定货币政策及市场参与者制定投资策略提供有关当前经济运行状况的基本信息。而在中长期,\yc 的水平利率反映的是市场对未来经济走势的理性预期。这也从另一方面说明了其代表市场利率的基准水平和未来变动方向,对\ts 的变动分析能够揭示众多有价值的经济信息,这些信息通过\yc 的形状、长短期利率的利差、长期均衡利率水平的变动等因素反映出来。例如\ts 作为先行商业周期指标(predictor)用于预测经济增长率\citep{harvey1993term}、 股票收益率\citep{campbell1987stock,cochrane2005bond}、 预期通货膨胀率
\citep{fama1990term}甚至汇率\citep{clarida2003out}。因此,对\ts 动态特征的研究显得尤为重要,特别是在固定收益证券(Fixed-Income)的投资领域,\ts 分析是一个重要的手段。\citeai{piazzesi2010affine} 在一篇\ts 的综述性文章中列举了至少四点原因,这包括预测未来短期利率、制定有效的中央银行货币政策、国债发行策略、衍生品定级与套期保值。

%长期以来,众多经济学家和金融学家对\ts 的形成机制投入了大量的精力与心血。对\ts 的研究经历了从传统的以解释其形成机理为主的定性分析到以统计拟合与预测为主的现代定量分析的发展过程,产生了大量的用于解释\ts 的形成机制与动态特征的\tsm{}。早期的\ts{}理论主要关注影响\ts 的形成机制的经济因素,如市场预期假说认为整个债券市场是统一的,不同到期期限的债券之间具有完全的替代性,从而每个债券购买者在不同期限的债券之间没有任何的个人特殊偏好,据此,\ts 的形态只取决于整个市场对未来利率的预期。后来发展的\tsm 主要包括两大类:(1)单因子模型中只含一个随机因子,这意味着收益率曲线上各点的随机因子完全相关,从而短期利率是影响债券收益率曲线的唯一状态变量;(2)而多因子模型则假定\ts 的随机动态演变过程是由几个因子共同推动的,这些因子可以是宏观经济的冲击或者\yc 本身的状况,如收益水平、\yc 的斜度与曲度。在实证分析方面,\tsm 主要采用现代统计拟合技术以及时间序列分析方法,把现有的收益率作为因变量,到期期限为自变量,对\yc 进行样本内拟合,并以估计的参数作为预测的依据。

然而,这些有关\ts 的理论与模型却有诸多的缺陷。一方面,要么\tsm 能够在理论上提供严谨的分析,却不能够满足实证要求,模型估计和预测的灵活性较差,难以反映实际的各种可能的\yc 和\ts 的动态特征;另一方面,一些\tsm 在实证方面取得了相对的成功,虽然能够对历史数据做良好的\yc 统计规律性分析,在样本内拟合与样本外预测都得到了一定的实证支持,但是缺乏坚实的经济理论基础,不能够提供对各种潜在变量的宏观经济学解释。

新近发展的以包含宏观基础变量与金融市场变量的无套利仿射利率期限结构模型(Macro-Finance Model)旨在为解释\ts 的动态特征提供理论支持,并在实际应用中提供了良好的样本内\yc 拟合以及在短期利率预测效果具有明显优势,尤其是\citeai{diebold2006forecasting}在\citeai{nelson1987parsimonious}提出的静态三因子\ts 的基础上拓展的动态~Nelson-Siegel~\tsm(以下简称为~DNS)。该模型不仅在对美国国债券收益率的样本内拟合十分成功,而且在预测方面也成绩斐然,尤其以预测中长期(~12~月以上)收益率更具明显优势,因此被很多从事经济政策制定和从事实证研究人员使用。\dns 也因此成为后来学界研究\tsm 的重要参考依据。然而,这个模型也同样面临着缺乏宏观经济学解释的尴尬:虽然\citeath{diebold2006forecasting}提供了三个因子的直观解释,即影响\ts 的因素可以归结为水平因子(Level Factor)、斜度因子(Slope Factor)以及曲度因子(Curvature Factor),且三个因子分别对应各种经济与金融市场的长期影响因素、短期影响因素和中期影响因素。然而,对于为何\dns 能够对中长期零息票国债券收益率上有如此强劲的预测能力,\citeath{diebold2006forecasting}及其后的拓展模型并未给出明确的答案\footnote{相关的文献资料将在~\ref{chap02-dev}~中详细讨论。}。这也是后来有关\tsm 发展以及本文的研究动机。

一个地区人口年龄结构指的是在某个特定时点上,由按不同年龄阶段划分的群体在总人群数的比例关系。与\ts 的形态相类似,\ds 在图像上也表现为一系列的动态曲线族。自工业革命以来,人口结构变化一直伴随着社会经济发展的整个过程,并对经济发展产生巨大的影响。其中,人口老龄化更是各国政府普遍重视的一个社会经济问题。早在~18~世纪,\citeai{malthus1798essay}便指出人口增长快于生产资料增长的自然规律将导致一个地区社会经济的崩溃。一些劳动经济学家、金融学家在~20~世纪~90~年代初就已经开始关注\ds 变化对地区经济、社会发展、金融市场产生的全局性、整体性、长期性的影响。他们试图通过严格的经济学理论假设与模型来研究\ds 与宏观经济变量和金融市场之间的关系。从横截性层面看(the cross-section),投资者在不同的年龄阶段有不同的投资偏好与风险喜恶态度,对财富的配置策略和金融资产的投资需求也随之变化,相应的,不同年龄阶段的人口群体在总人口的比例变化必然会影响整个宏观经济的运行状况、总消费、劳动供给、社会保险、养老金体系管理以及政府财政货币政策,例如一个正处于人口老龄化阶段的社会面临住房购买量下降及金融投资需求上升的压力,市场要求的风险贴水(risk premium)提高,这将导致证券收益率相应地降低\citep{bakshi1994baby}。\citeai{geanakoplos2004demography} 用一个世代交叠模型(Overlapping-Generation Model,OLG)探讨了美国\ds 与金融市场信贷之间的关系。根据资产投资的生命周期理论\citep{modigliani1954utility},一个典型的代表性消费者在年轻时由于收入不足以承担消费支出,在此阶段会通过资本市场进行借款,而当她进入中年阶段,其收入迅速上升并超过支出,此时她会倾向于将多余的收入进行投资(货币市场如银行机构,或者资本市场如购买证券),而等到该消费者退休之时,她会使用以前的储蓄收入以度晚年。这种典型的投资行为将通过社会加总的方式来决定货币市场与金融市场的均衡证券价格,继而决定均衡时的债券收益率和利率水平。由此可见,\ds 作为一个重要的影响因素与\ts 的长期均衡水平的波动存在着密切的关系。这类由\ds 的变动所引致的对\tsm 的影响只有在一个世代期限范畴内才能显现效果。

\section{研究目的及意义}{Purpose}

虽然一些金融学家早已开始关注\dsf 对金融市场特别是证券收益率的影响,如\citeai{yoo1994age}研究发现:人口年龄结构变动对不同风险的金融产品的投资需求有显著影响。最近\citeai{dellavigna2005attention}的一项研究也表明人口年龄结构变动会影响资本市场参与者对即期与远期市场的预测,对投资决策进而对资产价格产生影响。尽管如此,以往\tsm 的相关研究并没有直接考察\ds 对\ts 长期均衡水平波动的分析。这主要是由于传统\tsm 的研究方法以有关不同年龄阶段的人口群体的投资风险厌恶的同质性假设为基础,对\ts 的建模便无从考察随着人口年龄变化,整个金融市场的投资需求以及债券收益率是如何变动的。目前主要的宏观金融\tsm (MF-TSM)也只是关注少数几个宏观变量对\ts 的影响,并没有将\ds 作为一个长期影响因子纳入模型分析中。本文根据已有的\tsm 的理论文献,以\citeath{diebold2006forecasting} 拓展的\dns 为基础,提出了一个以人口因素驱动的动态\ts 拓展模型。一方面,该模型不仅能够理论上为理解\ts 的长期均衡水平波动提供很好的宏观经济解释,阐述了\dsf 在低频度层次上(low-frequency)影响未来利率的预期水平与变动方向,而且能够提供一个分析宏观经济基础变量与\ts 之间稳定关系的桥梁;另一方面,该模型不仅继承了\dns 的简约 特征(Principle of Parsimony),模型的估计参数较少,实证估计与拟合回归更加灵活,而且\dsf 本身所具有的的稳定的可预测性特征有利于增强对未来\ts 的水平变动的预测能力。

\section{研究框架}{Structure}
本文的研究框架如下:

第一章为``绪论'',介绍了本文的研究背景与研究意义。

第二章``文献回顾''部分则对现代利率期限结构模型研究的相关文献资料进行简要回顾及评述。本章介绍了\dns 的发展过程及其后的拓展模型,并讨论了在该模型的基础上引入\dsf 的分析。

第三章描述了本文的``模型设定及数据''。该部分将试图建立一个由人口因素驱动的\dns{},通过考察\ds 的动态特征,论证其反映的货币市场长期均衡时的利率水平的变动状况,即\dsf 决定了动态~Nelson-Siegel~模型中的长期影响因子。同时,该部分将对本文所使用的数据进行相应的说明。

第四章报告模型估计的``实证结果''。本章首先对美国国债券收益率数据做了样本内的拟合处理,随后对模型参数进行估计。

第五章分析了模型的样本外``预测结果''。

第六章为``结论'',对本文提出的模型及其实证结果进行简要总结,并提出了进一步研究的方向。

%\begin{itemize}
%\item[] 第一章为``绪论'',介绍了本文的研究背景与研究意义。
%\item[] 第二章``文献回顾''部分则对现代利率期限结构模型研究的相关文献资料进行简要回顾及评述。该部分介绍了\dns 的发展过程及其后的拓展模型,并讨论了在该模型的基础上引入\dsf 的分析。
%\end{itemize}




